\documentclass{article}
\usepackage[hyphens,spaces,obeyspaces]{url}
\usepackage{pgfplots}
\usepackage{tikz}
\usepackage{graphicx}
\graphicspath{ {./images/} }
\usepackage{amsmath}
\usepackage{amsthm}
\usepackage{amssymb}
\usepackage{mathabx}
\usepackage{amsfonts}
\usepackage{enumitem}
\graphicspath{ {./images/} }
\usetikzlibrary{shapes}
\usepgfplotslibrary{polar}
\usetikzlibrary{decorations.markings}
\usetikzlibrary{backgrounds}
\pgfplotsset{every axis/.append style={
                    axis x line=middle,    % put the x axis in the middle
                    axis y line=middle,    % put the y axis in the middle
                    axis line style={<->,color=blue}, % arrows on the axis
                    xlabel={$x$},          % default put x on x-axis
                    ylabel={$y$},          % default put y on y-axis
            }}
\newcommand{\numpy}{{\tt numpy}}    % tt font for numpy
\usepackage[utf8]{inputenc}

\newtheorem{theorem}{Theorem}
\newtheorem{lemma}[theorem]{Lemma}
\newtheorem{proposition}[theorem]{Proposition}
\newtheorem{corollary}[theorem]{Corollary}
\newtheorem{conjecture}{Conjecture}
\newtheorem{definition}{Definition}
\theoremstyle{remark}
\newtheorem{example}{Example}
\newtheorem{remark}[example]{Remark}

\title{Analysis Review}
\author{MinSeok Song}
\date{}

\begin{document}
\maketitle 

\subsection*{Sequences and series of functions}
\begin{itemize}
\item If $f$ is real-valued, then "complete" in uniform convergence sense (this can be useful for the proof, the reason $\mathbb{R}$ is complete is due to Heine-Borel).
\item (Thm 7.13, about when do we guarantee uniform convergence) It is useful to set $K_n=\{x\mid g_n (x)\geq \epsilon\}$ and show that $K_n=\phi$, in order to show uniform continuity (will be easy to use monotonicity, and extract finiteness for compactness).
\item Counterexample when $n$ is both on the coefficient and exponent. $f_n (x)=n^2 (1-x^2)^n$ or $f_n (x)=n(1-x^2)^n$. Note also that this is not monotonic in $n$.
\item why does $C(X)$ need to be bounded? We have an infinite supnorm.
\item clearly, $f_n(x)=\frac{\sin nx}{\sqrt n}$ gives the counterexample about derivatives. $\sin$ and $\cos$ are easy to play around when it comes to derivative.
\item Really nothing about the uniform convergence of derivatives. If we know that the derivatives converge uniformly, then we can actually find its value (thm 7.17). 
\item Way of proof: first we need to show uniform convergence of $f$. We manipulated inequality and used mean value theorem. Now defining $f$, realize that $f'(x)=\lim_{n\to\infty}f_n'(x)$ simply needs uniform convergence argument (switching of limits).
\item Nowhere differntiable continuous function: defined as infinite sum of scaled "absolute function." 
\item Equicontinuity is like uniform continuity for every $n$.
\item Ascoli-Arzela applies for when $K$ is compact, point-wise bounded, and equicontinuous. (then is uniformly bounded and exists uniformly convergent subsequence)
\item (Sort of) converse holds :we just need uniform convergence on a compact set. Then we can guarantee equicontinuity (uniform convergence is like a bridge between index, and compact set gives uniform continuity, essentially uniform continuity + uniform convergence = equicontinuity).
\item (Thm 7.23) pointwise bounded: for each point, we can extract convergent subsequence, countable set: extract diagonal elements. The key is to construct subsequence of subsequence, etc.
\item (Thm 7.25) Uniformly bounded: use compactness to use maximum, and point-wise boundedness. Convergent subsequence: divide into dense set (and use 7.23)/its complement (use density to divide into three).
\item Precompact set of $C(K)$ is equicontinuous(?, page 45).
\item 
\end{itemize}

\end{document}