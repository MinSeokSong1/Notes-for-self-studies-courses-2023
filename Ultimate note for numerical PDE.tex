\documentclass{article}
\usepackage[hyphens,spaces,obeyspaces]{url}
\usepackage{pgfplots}
\usepackage{tikz}
\usepackage{graphicx}
\graphicspath{ {./images/} }
\usepackage{amsmath}
\usepackage{amsthm}
\usepackage{amssymb}
\usepackage{mathabx}
\usepackage{amsfonts}
\usepackage{enumitem}
\graphicspath{ {./images/} }
\usetikzlibrary{shapes}
\usepgfplotslibrary{polar}
\usetikzlibrary{decorations.markings}
\usetikzlibrary{backgrounds}
\pgfplotsset{every axis/.append style={
                    axis x line=middle,    % put the x axis in the middle
                    axis y line=middle,    % put the y axis in the middle
                    axis line style={<->,color=blue}, % arrows on the axis
                    xlabel={$x$},          % default put x on x-axis
                    ylabel={$y$},          % default put y on y-axis
            }}
\newcommand{\numpy}{{\tt numpy}}    % tt font for numpy
\usepackage[utf8]{inputenc}

\newtheorem{theorem}{Theorem}
\newtheorem{lemma}[theorem]{Lemma}
\newtheorem{proposition}[theorem]{Proposition}
\newtheorem{corollary}[theorem]{Corollary}
\newtheorem{conjecture}{Conjecture}
\newtheorem{definition}{Definition}
\theoremstyle{remark}
\newtheorem{example}{Example}
\newtheorem{remark}[example]{Remark}

\title{Numerical PDE}
\author{MinSeok Song}
\date{}

\begin{document}
\maketitle 
\section*{Test}
\begin{itemize}
\item Note that eigenvector does not change, but eigenvalue changes. Write out the matrix formulation. Use educational guess for eigenvector components. The computation does not make sense. 

$$v_k=2ic_+ \sin k\xi$$ yields
$$\sin(n-2)\xi=\lambda_0\sin (n-1)\xi$$
but this does not give
$$c_+\sin(n\xi)=0$$.
\item Uniqueness tells us something about invertibility! 
\item Coefficients sum to zero$\to$ not good!
\item Brandle Hilbert why no polynomial?
\item 1. Bounded because $l\leq k$.

2. sublinear since norm.

3. Annihilation since subspace is a polynomial.

\item In $L^2$ sense, we get an order of $h^3$.

\item $\mid v(x)\mid^2\leq\lVert v'\rVert^2$.
\item We needed when we compare energy norm and the norm for $v'$.
\item How did we prove that the energy norm is an inner product?





\end{itemize}
\end{document}