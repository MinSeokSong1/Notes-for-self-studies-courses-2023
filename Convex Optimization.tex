\documentclass{article}

\usepackage{pgfplots}
\usepackage{tikz}
\usepackage{graphicx}
\graphicspath{ {./images/} }
\usepackage{amsmath}
\usepackage{amsthm}
\usepackage{amssymb}
\usepackage{mathabx}
\usepackage{amsfonts}
\usepackage{enumitem}
\graphicspath{ {./images/} }
\usetikzlibrary{shapes}
\usepgfplotslibrary{polar}
\usetikzlibrary{decorations.markings}
\usetikzlibrary{backgrounds}
\pgfplotsset{every axis/.append style={
                    axis x line=middle,    % put the x axis in the middle
                    axis y line=middle,    % put the y axis in the middle
                    axis line style={<->,color=blue}, % arrows on the axis
                    xlabel={$x$},          % default put x on x-axis
                    ylabel={$y$},          % default put y on y-axis
            }}
\newcommand{\numpy}{{\tt numpy}}    % tt font for numpy
\usepackage[utf8]{inputenc}

\newtheorem{theorem}{Theorem}
\newtheorem{lemma}[theorem]{Lemma}
\newtheorem{proposition}[theorem]{Proposition}
\newtheorem{corollary}[theorem]{Corollary}
\newtheorem{conjecture}{Conjecture}
\newtheorem{definition}{Definition}
\theoremstyle{remark}
\newtheorem{example}{Example}
\newtheorem{remark}[example]{Remark}

\title{Convex Optimization}
\author{MinSeok Song}
\date{}

\begin{document}
\maketitle
\begin{itemize}
\item Definition of Affine/convex sets/affine hull/affine dimension
\item relative interior(relint). We can modify the definition for the convex set. 
($https://en.wikipedia.org/wiki/Relative_interior$). Definition of relative boundary ($\bar C$\textbackslash relintC)

Intuition is clear by looking at 3D vs. 2D in page23 Ex2.2

Generalize convexity to function $p$ whose integral/infinite sum is one on the convex set, or even more expected value of $x$ with some underlying distribution.
\subsection*{Cones}
Cone is simply defined as $x\in C\to \theta\cdot x\in C$ for $\theta>0$.

Conic hull is the smallest convex cone.
\item
hyperplane is defined by $\{x\mid a^Tx=b\}$
Intuitively this can be expressed as $a^T(x-x_0)=0$.  $a$ is a normal vector and $a_0$ is an offset term. We can also define halfspace and the interior of half space.

\subsection*{Ellipsoid}Ellipsoid is given by $\{x\mid (x-x_c)^T P^{-1}(x-x_c)\leq 1\}$. We have $P^{-1}$ since the length should be on the denominator. The length of semi-axes is given by $\sqrt \lambda_i$'s.
\item We can also express as $\{x_c+Au\mid \lVert u\rVert_2\leq 1\}$ where $A$ is positive definite. When $A$ is positive semidefinite and $A$ is singular it is called a degenerate ellipsoid.
\item Norm cone is defined as $C=\{(x,t)\mid \lVert x\rVert \leq t\}$
\item simplex is defined by $\{\theta_0v_0+\dots+\theta_kv_k\mid \theta\succeq0, 1^T\theta=1\}$ where $v_i$'s are affinely independent ($v_{i+1}-v_i$'s are independent). Also called convex hull.
\item unit simplex $\{x\succeq 0, 1^Tx\leq 1\}$

\item probability simplex $x\succeq 0, 1^Tx=1$; can be expressed as polyhedron (idea is to make $AB=\begin{pmatrix}
I\\
0
\end{pmatrix}$).
\item Convex hull can be generalized to $\{\theta_1v_1+\dots+\theta_kv_k\mid\theta_1+\dots+\theta_m=1, \theta_i\geq 0, i=1,\cdots,k\}$ where $m\leq k$ (can be interpreted as convex hull + conic hull). Now this is characterization of a polyhedron.
\item \textbf{how do we show convexity? use definition, intersection, affine function, special functions...}
\item Positive semi definite is convex because it is intersection of half spaces (linear combination of elements in $A$).
\item (infinite) intersection of halfspaces $\Leftrightarrow$ closed convex
\item Image of convex under affine function is convex. Inverse image as well.
\item projection, Cartesian product, partial sums ($S=\{(x,y_1+y_2)\mid (x,y_1)\in S_1, (x,y_2)\in S_2\}$) of convex sets are convex.
\item (p38) inverse of the positive semidefinite cone? (p39) what is hyperbolic? inverse image confusing anyway
\subsection*{Perspective function}
\begin{definition}
Perspective function is $P:\mathbb{R}^{n+1}\to \mathbb{R}^n$ where $dom P=\mathbb{R}^n\times \mathbb{R}_{++}$ mapped by $P(z,t)=z/t$ 
(intuition: normalize by the last element and drop it, remember pinhole camera; this is why we have $[P(x),P(y)]=P([x,y])$).
\end{definition}
\item This explains why the image of convex is convex. The inverse image as well. (remember $\mu, 1-\mu$ is given by how much weight we ascribe to $t$ and $s$)
\begin{definition}
The linear-fractional (or projective) function is defined by $P\circ g$ where $g(x)=\begin{pmatrix}
A\\
C^T
\end{pmatrix}
x+
\begin{pmatrix}
b\\
d
\end{pmatrix}$, i.e. $f(x)=(Ax+b)/(c^Tx+b)$ with $dom f=\{x\mid c^Tx+b>0\}$
\end{definition}
\item (p42 check) what is the difference between z and v?
\end{itemize}
\subsection*{Generalized inequality}
\begin{itemize}
\item convex, closed. solid, pointed so that it has nice properties.
    \item Pointed means that pointed on zero. That is, if $x\in K,-x\in K$ then $x=0$. 
    \item what are the examples of proper cone?
    Symmetric matrix(positive semidefinite one is a proper cone) can be thought of as a cone.
    \item Next example restricts the dimension into $n-1$. Note we use "non negative" a lot here due to being a cone.
\end{itemize}
\subsection*{minimum/minimal}
\begin{itemize}
    \item Minimum: $S\subseteq x+K$
    \item Minimal: $(x-K)\cap S=\{x\}$
\end{itemize}
\subsection*{Separating hyperplane}
\begin{itemize}
    \item The specific choice of $a$ and $b$ so that it is orthogonal to $a$ and passes through the midpoint (using difference of squares formula).
    \item (page 49) Negative differentiation at zero $\to$ value at zero is larger than a little later than that.
\end{itemize}
\newpage
\section*{Digression}
\subsection*{Hahn-Banach}
\end{document}