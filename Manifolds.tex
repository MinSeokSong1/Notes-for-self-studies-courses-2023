\documentclass{article}
\usepackage[hyphens,spaces,obeyspaces]{url}
\usepackage{pgfplots}
\usepackage{tikz}
\usepackage{graphicx}
\graphicspath{ {./images/} }
\usepackage{amsmath}
\usepackage{amsthm}
\usepackage{amssymb}
\usepackage{mathabx}
\usepackage{amsfonts}
\usepackage{enumitem}
\graphicspath{ {./images/} }
\usetikzlibrary{shapes}
\usepgfplotslibrary{polar}
\usetikzlibrary{decorations.markings}
\usetikzlibrary{backgrounds}
\pgfplotsset{every axis/.append style={
                    axis x line=middle,    % put the x axis in the middle
                    axis y line=middle,    % put the y axis in the middle
                    axis line style={<->,color=blue}, % arrows on the axis
                    xlabel={$x$},          % default put x on x-axis
                    ylabel={$y$},          % default put y on y-axis
            }}
\newcommand{\numpy}{{\tt numpy}}    % tt font for numpy
\usepackage[utf8]{inputenc}

\newtheorem{theorem}{Theorem}
\newtheorem{lemma}[theorem]{Lemma}
\newtheorem{proposition}[theorem]{Proposition}
\newtheorem{corollary}[theorem]{Corollary}
\newtheorem{conjecture}{Conjecture}
\newtheorem{definition}{Definition}
\theoremstyle{remark}
\newtheorem{example}{Example}
\newtheorem{remark}[example]{Remark}

\title{Manifolds}
\author{MinSeok Song}
\date{}

\begin{document}
\maketitle                                                                                                                                                   
\begin{itemize}
\subsection*{Notion}
\item \textbf{Taylor's theorem} Use mean value theorem iteratively. In the proof, the argument of the error $f^{n}$ gets close to $x$ as $n$ gets higher but in fact the function $g$ (which we use MVT for) depends on $n$, so we can't say much.
\item In Tu's book, we can use the theorem iteratively on $g_i (x)$ to get higher terms. $g_i$ is basically coefficient of the highest term.
\item Definition of directional derivative (\textbf{can identify with simply $v$, proved by Taylor's theorem for surjectivity, since we can kill generality of $g_i$ by specifying a point $p$}), germ of $f$ at $p$, algebra over a field $K$ (example: $C^\infty_p$).
\item linear map $C^\infty_p\to\mathbb{R}$, satisfying Leibnitz rule = derivation at $p$
\item For vector fields we do not specify a point.
\item $Xf$ is natural derivatives.
\item Derivative of algebra is a map $f\to Xf$ satisfying $D(ab)=bDa+aDb$, nothing to do with real derivatives.
\begin{definition}
Covector $V^v$ is a linear function on $V$
\end{definition}
\item $V^v$ is a vector space, and the basis of $V$ corresponds to the basis of $V^v$ (coefficeints of $v\in V$).
\item $sgn(\sigma\circ \tau)=sgn(\sigma) sgn(\tau)$

\item Inversion: a pair $\sigma(i),\sigma(j)$ such that $i<j$ but $\sigma(i)>\sigma(j)$. We can see if odd or even by counting the number of inversion. This seems like an awfully ineffective method though. 
\item permutation calculated from right to left.
\item $k-linear$ functions ($V^k\to\mathbb{R}$), we have notion of symmmetric and alternating. We are interested in $A_k (V)$, alternating k-covectors.
\item Permutation action on multilinear functions.
\item In general, action $G$ on a set $X$, satisfying two axioms.
\item $S$ and $A$ in $Sf$ and $Af$ are simply sum of all permutations $\in S_k$.
\item tensor product of $k-linear$ and $l-linear$ is just natural $k+l-linear$ function.
\item Wedge product of alternating functions is defined by $f\wedge g=\frac 1{k!l!}A(f\otimes g)$. So it is a lienar function. Basically, dividing because duplicates arises (since $f$ and $g$ are alternating functions). Note also that $\sigma\tau=\tau\sigma$.
\item We can simplify using "(k,l)-shuffle" (basically ascending order).
\item Anticommutativity ($(-1)^{kl}$). This is why $f\wedge f=0$
\subsection*{Differential Form}
\item $(df)_p=<\cdot, f>:T_p(\mathbb{R}^n)\to \mathbb{R}$
\item Differential form $w=df:U\to\cup_{p\in U}T^* _p (\mathbb{R}^n)$
\item This is a vector space on $\mathbb{R}$, with basis $dx^i$.
\item Now we can justify $df=\sum\frac{\partial f}{\partial x^i}dx^i$ ($f$ smooth then $df$ also smooth).
\item Note that $A_0(T_p\mathbb{R}^n)=\mathbb{R}$. Subscript $0$ suggests the kind of argument we consider.
\item Differential form at point $p$ is linear function, and so we can define the Wedge product between them (note that $Af$ is alternating, and this wedge is anti-commutative and associative).
\item Notation: $\Omega^k (U)$ is a vector space of $C^\infty$ $k$-forms on $U$.
\item Graded algebra over a field is a form $\bigoplus_{k=0}^\infty A^k$ such that the multiplication map sends $A^k\times A^l$ to $A^{k+l}$. Differential form is the best example. Antiderivative of the graded algebra is a $K-$linear map $D:A\to A$ such that $D(ab)=(Da)b+(-1)^k aDb$
\item Three characterization of the exterior derivatives: antiderivation of degree 1, $d^2=0$, $(df)(X)=Xf$ for smooth function $f$ (prop 4.8).
\item Vector fields use subscript and differential forms use superscript.
\item $C^\infty (U)\to_{grad} VF \to_{curl} VF \to_{div} C^\infty (U)$
\item important fact: 1-form on $\mathbb{R}^3$ is exact iff it is closed.
\end{itemize}
\end{document}